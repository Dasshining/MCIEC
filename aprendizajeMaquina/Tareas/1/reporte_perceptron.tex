\documentclass{article}
\usepackage[utf8]{inputenc}
\usepackage{listings}
\usepackage{xcolor}
\usepackage{graphicx}
\usepackage{geometry}

\geometry{a4paper, margin=1in}

\title{Reporte: A01 El Perceptrón}
\author{Aldo Luna}
\date{\today}

\definecolor{codegreen}{rgb}{0,0.6,0}
\definecolor{codegray}{rgb}{0.5,0.5,0.5}
\definecolor{codepurple}{rgb}{0.58,0,0.82}
\definecolor{backcolour}{rgb}{0.95,0.95,0.92}

\lstdefinestyle{mystyle}{
    backgroundcolor=\color{backcolour},
    commentstyle=\color{codegreen},
    keywordstyle=\color{magenta},
    numberstyle=\tiny\color{codegray},
    stringstyle=\color{codepurple},
    basicstyle=\ttfamily\footnotesize,
    breakatwhitespace=false,
    breaklines=true,
    captionpos=b,
    keepspaces=true,
    numbers=left,
    numbersep=5pt,
    showspaces=false,
    showstringspaces=false,
    showtabs=false,
    tabsize=2
}

\lstset{style=mystyle,
    literate={á}{{\'a}}1
        {é}{{\'e}}1
        {í}{{\'i}}1
        {ó}{{\'o}}1
        {ú}{{\'u}}1
        {ñ}{{\~n}}1
        {Á}{{\'A}}1
        {É}{{\'E}}1
        {Í}{{\'I}}1
        {Ó}{{\'O}}1
        {Ú}{{\'U}}1
        {Ñ}{{\~N}}1
        {¿}{{\textquestiondown}}1
        {¡}{{\textexclamdown}}1
}

\begin{document}

\maketitle
\section{Perceptron}

\lstinputlisting[language=Python, caption=l01\_03\_perceptron.py]{l01_03_perceptron.py}

\subsection{Resultados Gráficos}

\begin{figure}[!ht]
    \centering
    % Asegúrate de que el nombre del archivo coincida con tu imagen en Tareas/1
    \includegraphics[width=0.6\textwidth]{or_plot.png} 
    \caption{Compuerta OR}
\end{figure}

\begin{figure}[!ht]
    \centering
    \includegraphics[width=0.6\textwidth]{and_plot.png} 
    \caption{Compuerta AND}
\end{figure}

\begin{figure}[!ht]
    \centering
    \includegraphics[width=0.6\textwidth]{xor_plot.png} 
    \caption{Compuerta XOR}
\end{figure}

\newpage
\section{Predicción de Sobrepeso}

\lstinputlisting[language=Python, caption=l01\_04\_predicción\_de\_sobrepeso.py]{l01_04_prediccion_de_sobrepeso.py}

\subsection{Resultados Gráficos}

\begin{figure}[!ht]
    \centering
    \includegraphics[width=0.6\textwidth]{sobrepeso_raw.png}
    \caption{Clasificación con datos originales}
\end{figure}

\begin{figure}[!ht]
    \centering
    \includegraphics[width=0.6\textwidth]{sobrepeso_norm.png}
    \caption{Clasificación con datos normalizados}
\end{figure}

\end{document}