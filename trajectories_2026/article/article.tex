% This is the English translation of the report in LLNCS format.
\documentclass[runningheads]{llncs}

% --- PACKAGES ---
\usepackage[T1]{fontenc}
\usepackage[utf8]{inputenc}
\usepackage[english]{babel} % Changed to English

\usepackage{graphicx}
\usepackage{amsmath}
\usepackage{booktabs}
\usepackage{listings}
\usepackage{algorithm}
\usepackage{algpseudocode}

\usepackage{color}
\usepackage[hidelinks]{hyperref}
\renewcommand\UrlFont{\color{blue}\rmfamily}

% --- LISTINGS CONFIGURATION ---
\definecolor{codegreen}{rgb}{0,0.6,0}
\definecolor{codegray}{rgb}{0.5,0.5,0.5}
\definecolor{codeblue}{rgb}{0,0,0.8}
\definecolor{codepurple}{rgb}{0.58,0,0.82}
\definecolor{backcolour}{rgb}{0.97,0.97,0.97}

\lstdefinestyle{mystyle}{
    backgroundcolor=\color{backcolour},   
    commentstyle=\color{codegreen},
    keywordstyle=\color{codeblue},
    stringstyle=\color{codepurple},
    basicstyle=\footnotesize\ttfamily,
    breakatwhitespace=false,         
    breaklines=true,                 
    captionpos=b,                    
    keepspaces=true,                 
    showspaces=false,                
    showstringspaces=false,
    showtabs=false,                  
    tabsize=2
}
\lstset{style=mystyle}

\begin{document}

% --- HEADER (Title, Author, Institute) ---
\title{Optimization of Trunk Public Transport Routes in Guadalajara}
% \titlerunning{Route Optimization in Guadalajara} % Optional if title is too long

\author{Rafael Aldo Hernández Luna}
\authorrunning{R. A. Hernández Luna}

\institute{University Center of Exact Sciences and Engineering (CUCEI),\and
University of Guadalajara, Guadalajara, Mexico\\
\email{Project Report - Intelligent Systems}}
\url{https://www.cucei.udg.mx/}
\maketitle

% --- ABSTRACT ---
\begin{abstract}
This report presents the development of an intelligent system for the
optimization of urban bus routes using genetic algorithms and graph theory. The
problem is addressed by modeling the city as a network of nodes and edges,
seeking to minimize a hybrid cost function that weighs both the service
provider's operational costs and the users' travel time. A simulation was
implemented using a passenger demand matrix, identifying \textbf{hotspots} or
high-traffic points. The results demonstrate the algorithm's capability to
generate routes that reduce passenger time, although the complexities of
implementation in a real urban environment due to legal and zoning factors are
highlighted.

\keywords{Genetic Algorithms \and Route Optimization \and Public Transport \and Graph Theory \and Guadalajara.}
\end{abstract}

% --- SECTIONS ---
\section{Introduction}
Public transportation systems serve as the backbone of sustainable urban
development, facilitating economic growth and social equity by providing
essential mobility to the population. As cities expand, the challenge of
maintaining an efficient, accessible, and sustainable transport network becomes
increasingly complex. Global studies emphasize that effective public transport
must not only reduce reliance on private vehicles to lower emissions but also
ensure equitable access to opportunities for all socioeconomic groups [1].
However, in many developing metropolitan areas, these systems struggle to meet
the dual goals of sustainability and efficiency. Reviews of sustainable
transportation concepts highlight that despite the clear environmental and
social benefits of public transit, the practical implementation often faces
significant hurdles related to planning, funding, and adapting to rapid urban
sprawl [1].

In the context of Mexico, and specifically the Guadalajara Metropolitan Area
(GMA), these challenges are acute. The public transport system, including the
main bus lines and the SITEUR (Sistema de Tren Eléctrico Urbano) network, plays
a critical role in the daily lives of millions. However, recent analyses
indicate that the current infrastructure is often insufficient to cover the
"transport social needs" of the population effectively. For instance, studies
targeting Sustainable Development Goal 11.2 in Guadalajara have found that
approximately 50.3\% of inhabitants reside in areas with very high social
transport needs, yet the coverage remains uneven [2]. Furthermore, historical
reviews of public transport in Guadalajara from 1960 to 2020 reveal that the
system's efficiency has been hampered by issues related to ownership structure,
fare policies, and a lack of focus on social justice in route planning [3]. The
result is a system often characterized by slow transportation times, poor
accessibility, and insufficient coverage, which directly diminishes the quality
of life for a significant percentage of inhabitants [3].

Existing literature has extensively documented the sustainability deficits [1]
and social exclusion caused by the current transport layout in Guadalajara
[2][3], there is a lack of research focusing on the dynamic and data-driven
optimization of these specific routes that this research aims to tackle. The
primary objective is to utilize bio-inspired optimization algorithms and graph
theory to simulate the passenger flow (affluence) through the city's transport
network. By modeling the city as a graph where nodes represent stops and edges
represent routes, this research aims to determine if the current network
configuration is fit to cover the mobility demand and to demonstrate how
rearranging routes can improve efficiency, accessibility, and ultimately, the
quality of life for users.

\section{Methodology}

\subsection{Network Modeling and Representation} 
To model the complex street network of the Guadalajara Metropolitan Area, this
study utilizes graph theory, specifically representing the city as a Weighted
Directed Multigraph. The foundational graph was extracted from real-world
OpenStreetMap (OSM) data. In this model, nodes represent street intersections or
specific locations (such as transit stops), while directed edges represent the
navigable street segments connecting them.

A critical component of this network is the assignment of weights to the edges,
which symbolize the travel time (impedance) required to traverse a segment.
Travel time is calculated as a function of the street segment length and an
estimated traversal speed. To better reflect real-world urban dynamics, a
dynamic traffic noise multiplier was introduced to the edge weights, simulating
variable congestion patterns across the network.

Simultaneously, passenger affluence is modeled using a demand matrix. This
matrix maps the origin-destination (O-D) flow of passengers across the network.
To represent urban focal points—such as commercial centers, universities, and
historical districts—specific high-traffic nodes were designated as "hotspots".
The demand matrix artificially inflates the passenger volume traveling to and
from these hotspots, providing a realistic baseline of public transport demand.

\subsection{Cost Function Formulation}
In order to assess alternative public bus routes that enhance time efficiency
and accessibility, a compound cost function was formulated. This function
evaluates both the operational expenses of the transit provider and the travel
time incurred by users traversing from various origins ($A$) to destinations
($B$) within the multigraph. The objective function $J$ is defined to be
minimized:
\begin{equation}
    J = (C_{operator} \cdot \beta) + (C_{user} \cdot \alpha)
\end{equation}
The operator cost ($C_{operator}$) is determined by the total temporal duration
of the proposed routes and an associated operational expense rate (e.g.,
maintenance, fuel, and wages). It is approximated as:
\begin{equation}
    C_{operator} \approx \text{Route Duration} \times \text{Cost per minute}
\end{equation}
Conversely, the user cost ($C_{user}$) quantifies the total time investment of
the passengers. It is calculated by aggregating the travel time of all
passengers across their respective route segments:
\begin{equation}
    C_{user} = \sum (\text{Passengers in segment} \times \text{Travel time})
\end{equation}
The parameters $\alpha$ and $\beta$ serve as scaling factors to steer the
algorithm's optimization tendency:
\begin{itemize}
    \item Prioritizing $\beta$: Favors shorter routes with lower operational costs,
    potentially at the expense of network coverage.
    
    \item Prioritizing $\alpha$: Favors longer, more comprehensive routes that provide
    high coverage and reduce user travel time.
\end{itemize}
The computational evaluation of this fitness function requires calculating the
shortest paths for all passengers using the proposed transit network. Below is
the pseudo-code illustrating the evaluation process:

\begin{algorithm}
\caption{Compound Cost Function Evaluation}
\begin{algorithmic}[1]
\Require Candidate routes ($R$), City Graph ($G$), Passenger Demand Matrix ($D$), Hotspots ($H$)
\Ensure Total Cost ($J$)
\State $OperatorCost \gets 0$
\State $UserCost \gets 0$
\State $TransitGraph \gets \text{Copy}(G)$
\Statex \textbf{// Step 1: Calculate Operator Cost \& Update Transit Graph}
\ForAll{route in $R$}
    \ForAll{segment $(u, v)$ in route}
        \State $TravelTime \gets \text{GetEdgeWeight}(G, u, v)$
        \State $OperatorCost \gets OperatorCost + (TravelTime \times BusSpeedMultiplier) \times CostPerMinute$
        \State $\text{UpdateEdgeWeight}(TransitGraph, u, v, TravelTime \times BusSpeedMultiplier)$
    \EndFor
\EndFor
\Statex \textbf{// Step 2: Calculate User Cost via Shortest Paths}
\ForAll{source in $H$}
    \If{source has departing passengers in $D$}
        \State $ShortestPaths \gets \text{Dijkstra}(TransitGraph, source)$
        \ForAll{destination, passenger\_count in $D[source]$}
            \If{destination is reachable in $ShortestPaths$}
                \State $UserCost \gets UserCost + (passenger\_count \times ShortestPaths[destination])$
            \Else
                \State $UserCost \gets UserCost + (passenger\_count \times UnservedPenalty)$
            \EndIf
        \EndFor
    \EndIf
\EndFor
\Statex \textbf{// Step 3: Apply Tendency Parameters}
\State $J \gets (OperatorCost \times \beta) + (UserCost \times \alpha)$
\State \Return $J$
\end{algorithmic}
\end{algorithm}

\newpage
\subsection{Bio-Inspired Optimization via Genetic Algorithm} 
Due to the NP-Hard nature of the transit routing problem and the
non-differentiable characteristics of the proposed cost function over a massive
urban graph, traditional deterministic optimization methods are computationally
prohibitive. Therefore, this study employs a Genetic Algorithm (GA) to navigate
the vast solution space. GAs have been extensively validated in literature as
highly robust mechanisms for solving multi-objective transport routing problems
where local optima trap standard algorithms.

The GA implementation relies on a specific chromosomal representation. Each
candidate solution (individual) is represented as an array of lists, where each
list defines a specific bus route. The elements within these lists are integer
node IDs corresponding to junctions in the city graph. Because nodes represent
intersections with multiple adjacent connections, the algorithm permits
flexibility in route generation; routes are constructed via random walks through
the graph's topology rather than strict linear geographic constraints.

The evolutionary process is driven by the following operators:

\begin{itemize}
    \item Initialization: The initial population is generated by creating random
    viable paths (random walks) originating from arbitrary nodes. 
    
    \item Fitness Evaluation: The compound cost function () is applied. Because
    the GA is designed to maximize fitness, the cost is inverted (e.g., ). 

    \item Selection: A tournament selection method is utilized to choose parent
    chromosomes, favoring individuals with lower overall network costs. 

    \item Crossover: A single-point crossover mechanism swaps entire routes
    between two parent solutions to generate offspring. 

    \item Mutation: To maintain genetic diversity and prevent premature
    convergence, offspring undergo mutation with a specified probability.
    Mutation operators include extending a route to an adjacent node, trimming
    the final stop, or entirely regenerating a route within the chromosome. 
\end{itemize}

The algorithm iteratively evolves the population, continuously plotting the most
fit routes against the spatial data of Guadalajara until a maximum generation
threshold is reached, yielding the optimized transit network configuration.

\subsection{Urban Graph and Demand Parameters}
To ensure a faithful representation of the metropolitan area's reality, the
following parameters were used:

\begin{table}
    \centering
    \caption{System Parameters} \label{tab:params}
    \begin{tabular}{@{}llp{6cm}@{}}
        \toprule
        Parameter & Value & Description \\
        \midrule
        Graph Source & OpenStreetMap & Network data extracted for Guadalajara,
        Mexico (Drive network type).\\
        Base Travel Speed & ~24 km/h & (Average per MIDE Jalisco) \\
        Traffic Noise & 1.0 - 1.5x & Stochastic multiplier applied to edge 
        weights to simulate variable traffic congestion.\\
        Hotspots & 8 Locations & Key coordinates (e.g., Historic
        Center, Zapopan, CUCEI) used to generate synthetic high-demand trips.\\
        \bottomrule
    \end{tabular}
\end{table}

\section{Results}
The simulation was executed on a graph representing an approximation of the city
(due to the computational complexity of processing the full network with OSMnx).

\subsection{Algorithm Evolution}
The behavior of the algorithm across generations is shown below.

\begin{figure}
    \centering
    \includegraphics[width=0.8\textwidth]{Figure_1.png}
    \caption{Fitness Evolution over 50 generations. An improvement in the cost function is observed, stabilizing near generation 43.}
    \label{fig:fitness_graph}
\end{figure}

The algorithm demonstrated convergence capability. In generation 43, a
\textit{Current Fitness} above 37.9 was reached, after which a plateau in
fitness values can be seen, indicating that the genetic algorithm successfully
explored the non-differentiable search space.

\subsection{Route Visualization}
The resulting routes showed an interesting tendency to overlap with existing
infrastructure lines (Light Rail and Macrobús).

\begin{figure}
    \centering
    \includegraphics[width=0.8\textwidth]{gen_037.png}
    \caption{Visualization of the 10 optimized routes on the city graph.}
    \label{fig:gen_037}
\end{figure}

\section{Conclusions}
The genetic algorithm proves to function adequately in finding routes that
minimize time costs for passengers. However, transitioning this theoretical
model to a real urban development presents additional challenges that cannot be
ignored. Although mathematically optimal under the current cost function, the
viability of these routes cannot be ensured without considering:

\begin{enumerate}
    \item \textbf{Legal and Traffic Framework:} Turn restrictions, street
    directions, and public transport regulations.
    \item \textbf{Zoning:} The distribution of commercial versus residential
    zones affects peak hours, something a static demand matrix does not fully
    capture.
    \item \textbf{Population Density:} Factors specific to Guadalajara that
    would add greater complexity to modeling edge weights.
\end{enumerate}

As future work, it is proposed to revisit the objective function to optimize the
processing of full graphs using tools like OSMnx and to refine edge weights with
real-time traffic data.

% --- BIBLIOGRAPHY ---
% Adapted to splncs04 style suggested in samplepaper.tex
\begin{thebibliography}{8}

\bibitem{consumidor}
El Poder del Consumidor: Urge transformación de transporte público en
Guadalajara y Monterrey: usuarios pierden tiempo, dinero y calidad de vida.
Retrieved from:
\url{https://elpoderdelconsumidor.org/2015/07/urge-transformacion-de-transporte-publico-en-guadalajara-y-monterrey-usuarios-pierden-tiempo-dinero-y-calidad-de-vida}
(2015)

\bibitem{mide}
MIDE Jalisco: Indicadores de Desarrollo. Retrieved from:
\url{https://mide.jalisco.gob.mx} (n.d.)

\bibitem{boeing2025}
Boeing, G.: Modeling and Analyzing Urban Networks and Amenities with OSMnx.
Geographical Analysis. doi:10.1111/gean.70009 (2025)

\bibitem{wiki_rutas}
Wikipedia: Anexo: Rutas de transporte en Jalisco. Retrieved from:
\url{https://es.wikipedia.org/wiki/Anexo:Rutas_de_transporte_en_Jalisco} (n.d.)

\end{thebibliography}

\end{document}